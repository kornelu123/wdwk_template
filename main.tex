\documentclass{article}

\usepackage[utf8]{inputenc}
\usepackage{titlesec}
\usepackage[T1]{fontenc}
\usepackage{xcolor}
\usepackage{listings}
\usepackage{color}
\usepackage{tikz}
\usepackage{graphicx}

\definecolor{mGreen}{rgb}{0,0.6,0}
\definecolor{mGray}{rgb}{0.5,0.5,0.5}
\definecolor{mPurple}{rgb}{0.58,0,0.82}
\definecolor{backgroundColour}{rgb}{0.95,0.95,0.92}

\lstdefinestyle{CStyle}{
  backgroundcolor=\color{backgroundColour},   
  commentstyle=\color{mGreen},
  keywordstyle=\color{magenta},
  numberstyle=\tiny\color{mGray},
  stringstyle=\color{mPurple},
  basicstyle=\footnotesize,
  breakatwhitespace=false,         
  breaklines=true,                 
  captionpos=b,                    
  keepspaces=true,                 
  numbers=left,                    
  numbersep=5pt,                  
  showspaces=false,                
  showstringspaces=false,
  showtabs=false,                  
  tabsize=2,
  language=C
}

\lstdefinestyle{customasm}{
  backgroundcolor=\color{backgroundColour},   
  commentstyle=\color{mGreen},
  keywordstyle=\color{magenta},
  numberstyle=\tiny\color{mGray},
  stringstyle=\color{mPurple},
  basicstyle=\footnotesize,
  breakatwhitespace=false,         
  breaklines=true,                 
  captionpos=b,                    
  keepspaces=true,                 
  numbers=left,                    
  numbersep=5pt,                  
  showspaces=false,                
  showstringspaces=false,
  showtabs=false,                  
  tabsize=2,
  language=[x86masm]Assembler,
}

\lstdefinestyle{bash}{
  backgroundcolor=\color{backgroundColour},   
  commentstyle=\color{mGreen},
  keywordstyle=\color{magenta},
  numberstyle=\tiny\color{mGray},
  stringstyle=\color{mPurple},
  basicstyle=\footnotesize,
  breakatwhitespace=false,         
  breaklines=true,                 
  captionpos=b,                    
  keepspaces=true,                 
  numbers=left,                    
  numbersep=5pt,                  
  showspaces=false,                
  showstringspaces=false,
  showtabs=false,                  
  tabsize=2,
  language=bash
}

\lstdefinestyle{make}{
  backgroundcolor=\color{backgroundColour},   
  commentstyle=\color{mGreen},
  keywordstyle=\color{magenta},
  numberstyle=\tiny\color{mGray},
  stringstyle=\color{mPurple},
  basicstyle=\footnotesize,
  breakatwhitespace=false,         
  breaklines=true,                 
  captionpos=b,                    
  keepspaces=true,                 
  numbers=left,                    
  numbersep=5pt,                  
  showspaces=false,                
  showstringspaces=false,
  showtabs=false,                  
  tabsize=2,
  language=make
}

\titleformat{\section}{\normalfont\Large\bfseries}{\thesection}{1em}{}
\titleformat{\subsection}{\normalfont\large\bfseries}{\thesubsection}{1em}{}

\renewcommand{\contentsname}{Spis treści}

\begin{document}

\title{Sprawozdanie z laboratorium przedmiotu "Wstęp do wysokowydajnych komputerów"
\large{Badanie wydajności programów}
}
\author{Kornel Uriasz 272967 ITE 2022/2026}
\date{26 Kwietnia 2024}
\maketitle
\newpage

\tableofcontents

\newpage

\section{Cel ćwiczenia}

Celem ćwiczenia było zapoznanie się z metodami pomiaru długości trwania programów 
napisanych w języku asemblera na platformie \verb|Linux/x86|.
Zgodnie z przekazanymi wymaganiami w ramach laboratoriów mieliśmy wykonać następujące zagadnienia:

\begin{enumerate}
  \item Przetworzenie kodu z poprzednich laboratoriów, generującego n-tą liczbe fibonacciego aby był zgodny z \verb|ABI języka C| i zwracał
    ilość taktowań potrzebną na wykonanie programu, poprzez użycie instrukcji \verb|sdtrc|.
  \item Wielokrotny pomiar czasu wykonania programu dla różnych argumentów wejściowych, ze sprawdzaniem poprawności działań.
  \item Optymalizacja kodu na podstawie wyników pomiarów, z użyciem programu \verb|perf|.
\end{enumerate}

Dwa pierwsze zagadnienia zostały zrealizowane na zajęciach, zgodnie z wymaganiami prowadzącego poprzez stworzenie
funkcji głównej programu w pliku \verb|main.c|, oraz przerobienie kodu z ostatnich zajęć w pliku \verb|fibonacci.s|.
Program był wywoływany 25 razy przez plik skryptowy języka powłoki systemowej \verb|test.sh|, dane wejściowe były przekazywane poprzez
przekierowanie plików testowych (f3000.in, f10000.in) do standardowego strumienia wejścia. Dane wyjściowe były przez skrypt porównywane
z wzorcowymi danami wyjściowymi programem \verb|diff|. Czasy wykonania funkcji przez program głowny zostały zapisane do plików times\_fxxx.txt.

Podczas zajęć został też zrealizowany częściowo podpunkt 3. Częściowe wykonanie polegało na wygenerowaniu pliku \verb|perf.data| przy użyciu
komendy \verb|perf record xxxx|.

Zadanie było realizowane na prywatnym laptopie, model procesora : \verb|Intel Core i5-1035G1 @ 1.00GHz|.
Została użyta maszyna wirtualna xubuntu dostępna na eportalu, zasoby to : 4 wątki, 4GB pamięci RAM.
\newpage
\section{Przebieg ćwiczenia}
